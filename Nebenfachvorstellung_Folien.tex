\documentclass[11pt]{beamer}
\usetheme{Dresden}
\usepackage[utf8]{inputenc}
\usepackage[ngerman]{babel}
\usepackage[T1]{fontenc}
\usepackage{amsmath}
\usepackage{amsfonts}
\usepackage{amssymb}
\usepackage{pifont}
\usepackage{color, colortbl}
\newcommand{\cmark}{\ding{51}}
\newcommand{\xmark}{\ding{55}}
\usepackage{fontawesome}

% \usepackage{emoji}


\usepackage[backend=biber,
style=alphabetic,
]{biblatex}



\usecolortheme{beaver}


\title[Anwendungsfach / Nebenfach: Maschinenbau] % (optional, only for long titles)
{Maschinenbau}
\author {Pina \faHandPeaceO}


\date[WiSe2021] % (optional)
{Anwendungsfach / Nebenfach}
\subject{Computer Science}

\beamertemplatenavigationsymbolsempty


\begin{document}






\begin{frame}
	\titlepage
\end{frame}













\begin{frame}{Nebenfach und Anwendungsfach}


\begin{columns}[T] % align columns
\begin{column}{.48\textwidth}
\color{red}

Studiengang Kerninformatik: \\
\begin{itemize}
\item Nebenfach mit 20 CP
\end{itemize}


\end{column}%
\hfill%
\begin{column}{.48\textwidth}
\color{blue}


Angewandte Informatik: \\
\begin{itemize}
\item Anwendungsfach mit 36 CP
\end{itemize}


\end{column}%
\end{columns}
\end{frame}









\begin{frame}{Pflichtmodule}

\begin{tabular}{l l}

Modul & CP \\
& \\
Fertigungslehre & 3CP \\
Technisches Zeichnen & 3CP \\
Maschinenelemente & 4CP \\

\end{tabular}


\end{frame}











\end{document}